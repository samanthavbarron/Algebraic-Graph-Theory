%------------------------------------------------------------------------------
% Beginning of journal.tex
%------------------------------------------------------------------------------
%
% AMS-LaTeX version 2 sample file for journals, based on amsart.cls.
%
%        ***     DO NOT USE THIS FILE AS A STARTER.      ***
%        ***  USE THE JOURNAL-SPECIFIC *.TEMPLATE FILE.  ***
%
% Replace amsart by the documentclass for the target journal, e.g., tran-l.
%
\documentclass{amsart}

%     If your article includes graphics, uncomment this command.
\usepackage{graphicx}
\usepackage{lipsum}
\usepackage{listings}
\usepackage{import}
\usepackage[final]{pdfpages}


\import{./}{defns.tex}

\begin{document}

\title{Algebraic Graph Theory}

%    Information for second author
\author{George Barron}
\email{george.samuel.barron@gmail.com}

\date{\today}

\keywords{Algebraic graph theory, Graph theory, Algebra, Mathematica}

\begin{abstract}
    Let $G$ be a graph and let $L$ be the Laplacian matrix of this graph. Associated with each graph is the Smith Normal Form of $L$, which we explore with computations in Mathematica. Provided are examples of said computations that lead to several conjectures, as well as a package that allows for efficient and intuitive computations.
\end{abstract}

\maketitle

\section{Introduction}

    \import{content/}{introduction.tex}
    
\section{Families of Graphs}
    
    In each of the following families of graphs, computational evidence is provided to motivate these conjectures. This data is available in the deliverables section of this report.
    
    \subsection{Trees}
        \import{content/}{tree.tex}
        
    \subsection{Cyclic Graphs}
        \import{content/}{cyclic.tex}
        
    \subsection{Complete Graphs}
        \import{content/}{complete.tex}
        
    \subsection{Path Graphs}
        \import{content/}{path.tex}
        
    \subsection{Grid (Lattice) Graphs}
        \import{content/}{grid.tex}
        
    \subsection{Regular Graphs}
        \import{content/}{regular.tex}

\section{Graph Products}

    \subsection{Gprod}
        \import{content/}{gprod.tex}
        
        \import{content/}{gprodcomplete.tex}
    
    \subsection{G2prod}
        \import{content/}{g2prod.tex}
        
        \import{content/}{g2prodcomplete.tex}
    
\section{Deliverables}

    \import{content/}{listings.tex}

\end{document}

%------------------------------------------------------------------------------
% End of journal.tex
%------------------------------------------------------------------------------
